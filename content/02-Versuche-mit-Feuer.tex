\newpage
\chapter{Versuche mit Feuer}

\section{Abgeschnittene Flamme}
% es Fehlen noch Bilder

Dieser Versuch ist \cite{Physikanten} entnommen.
Mit ihm lässt sich die Wärmeleitung von Metallen zeigen, sowie das Funktionsprinzip von Grubenlampen erklären.

\subsection{Material}

Benötigt wird ein Drahtgewebe, ein Bunsenbrenner und ein Feuerzeug.

\subsection{Sicherheitshinweise}

Bei diesem Versuch ist darauf zu achten, dass er nur in einem gut belüfteten Raum stattfindet.
Idealerweise wird der Versuch draußen durchgeführt.\\
\\
Beim festhalten des Drahtgewebes ist darauf zu achten, dass man es an den kurzen Seiten festhält, sodass der Abstand zwischen Hand und Flamme größt möglich ist.
Beim festhalten braucht man keine Angst vor Verbrennungen durch das Gitter zu haben, da das Aluminium die Wärme schnell genug abführt.

\subsection{Durchführung}

Es wird überprüft, ob der Bunsenbrenner richtig auf der Gaskatusche aufgeschraubt ist und beide Ventile des Brenners zugedreht sind.
Nun kann das Drahtgewebe auf den Bunsenbrenner gelegt werden.
Das Gewebe muss mit einer Hand fest gehalten werden.
Jetzt kann - mit der anderen Hand - der Gashahn des Brenners aufgedreht und anschließend das Gas mit dem Feuerzeug entzündet werden.
Zuletzt muss noch das zweite Ventil des Brenners aufgedreht werden, sodass sich die zuvor gelbe Flamme nun kräftig blau färbt.
Nun ist noch gut zehn Sekunden zu warten und das Gitter unbewegt zu halten.
Nach dieser Zeit kann das Gitter langsam angehoben werden.

\subsection{Beobachtungen}

Man verbrennt sich die Finger nicht, wenn man das Gitter festhält.\\
Hebt man das Gitter hoch, so brennt die Flamme nur über dem Gitter, nicht jedoch unterhalb.

\subsection{Erklärung}

Da Metalle gute Wärmeleiter sind, leitet das Drahtgewebe die Wärme schnell großflächig ab und die Zündtemperatur des Gases wird unterhalb des Netzes nicht mehr erreicht.\\
Grubenlampen funktionieren nach dem selben Prinzip.

\section{Der Geysir}

\section{Rubenssches Flammenrohr}
