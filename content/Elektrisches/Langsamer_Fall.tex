\section{Der langsame Fall}

Dieser Versuch ist \cite{Physikanten} entnommen.
Hier wird das Funktionsprinzip von Wirbelstrombremsen, die im ICE Anwendung finden, erklärt.

\subsection{Material}

\begin{itemize}
	\item Aluschiene
	\item starken Magneten
	\item evtl. eine Münze
\end{itemize}

\subsection{Durchführung}

Man nehme die Aluschiene und halte sie sehr steil, sodass der Magnet hineingelegt werden kann.
Nun wird der Magnet hinein gelegt und sein \gg Fall \ll beobachtet.\\
\\
Es kann jeder diesen Versuch durchführen und die Schiene unterschiedlich stark geneigt werden.
Man kann versuchen heraus zu finden, ab welchem Winkel sich der Magnet zu bewegen beginnt.\\
Rutscht eine Münze auch erst bei der selben Neigung wie der Magnet?

\subsection{Beobachtungen}

Der Magnet rutscht die Schiene sehr langsam hinunter.
Zum Vergleich kann man einen in die Schiene passenden Gegenstand ebenfalls die Schiene hinunterrutschen lassen.
Hier stellt man fest, dass der Magnet wesentlich langsamer ist.

\subsection{Erklärung}

Der Magnet induziert in der Aluschiene kleine Kreisströme (Wirbelströme).
Da induzierter Strom immer der Ursache entgegen wirkt und durch einen Strom immer ein Magnetfeld entsteht, entsteht also ein dem Magnetfeld des Magneten entgegengerichtetes Magnetfeld.
Da sich entgegengerichtete Magnetfelder anziehen, wird der Magnet ausgebremst.\\
\\
Nach dem selben Prinzip funktionieren auch die Bremsen des ICE (Wirbelstrombremsen).
